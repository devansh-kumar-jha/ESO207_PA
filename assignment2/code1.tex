\documentclass[20pt]{article}
\usepackage{amsmath}
\usepackage{graphicx}
\usepackage{hyperref}
\usepackage[utf8]{inputenc}
\usepackage{polynom}
\usepackage[ruled, lined, linesnumbered, commentsnumbered, longend]{algorithm2e}
\usepackage{xcolor}

\title{ESO207 Question 1}
\author{Devansh Kumar Jha}
\date{2021–10-15}
\begin{document}
\maketitle


\section{Programming Problem 1}

\subsection{Data Structure Usage}
Two-Three tree ADT controlled using a class "twth".

\subsection{Strategy Used}

\subsection{Pseudo Code}
These are the variables used - 
\begin{itemize}
\item \textbf{th1}
\item \textbf{th2}
\end{itemize}
\begin{algorithm}

	\SetKwFunction{isOddNumber}{isOddNumber}
	\SetKwInOut{KwIn}{Input}
	\SetKwInOut{KwOut}{Output}
    
    \If{$th1.get() \rightarrow type == 0$} {
        $th.set(th2.get())$ \\
        $return (th)$ \\
    }
    \If{$th2.get() \rightarrow type == 0$} {
        $th.set(th1.get())$ \\
        $return (th)$ \\
    }
    $h1=1,h2=1$ \\
    $temp1=th1.get()$ \\
    $temp2=th2.get()$
    \While{$temp1 \rightarrow type>2$}{
        $h1++$ \\
        \eIf{$temp1 \rightarrow right==NULL$}{
            $temp1=temp1 \rightarrow middle$
        } {
            $temp1=temp1 \rightarrow right$
        }
    }
    \While{$temp2 \rightarrow type>2$}{
        $h2++$ \\
        $temp2=temp2 \rightarrow left$
    }
    \eIf{h1==h2}{
        $th.set(twonode(temp2 \rightarrow d1,th1.get(),th2.get()))$
    }
     {
         \eIf{$h1>h2$}{
        \For{$i \leftarrow 0$ \KwTo $h2$} {
        $temp1=temp1 \rightarrow parent$ \\
        $th1.insert(th2.get(),temp1,temp2->d1,2)$ \\
        $th.set(th1.get())$
        }
    }
    
     {
        \For{$i \leftarrow 0$ \KwTo $h1$} {
        $temp2=temp2 \rightarrow parent$ \\
        $th2.insert(th1.get(),temp2,th2.min(),1)$ \\
        $th.set(th2.get())$
        }
    }
     }
\end{algorithm}

\begin{algorithm}
    $ret\_insert\_part* p=new ret\_insert\_part$
    \eIf{$pos \rightarrow type==3$}{
        $pos \rightarrow type==4$ \\
        \eIf{$type==1$}{
            $pos \rightarrow right=pos \rightarrow middle$ \\
            $pos \rightarrow middle=pos \rightarrow left$ \\
            $pos \rightarrow left=node$ \\
            $pos \rightarrow d2=pos \rightarrow d1$ \\
            $pos \rightarrow d1=m$
        } {
            $pos \rightarrow d2=m$ \\
            $pos \rightarrow right=node$
        }
        $node \rightarrow parent=pos$ \\
        $p \rightarrow n1=NULL$ \\
        $p \rightarrow n2=NULL$ \\
        $p \rightarrow m= -1$ 

    } {
        $pos \rightarrow type=3$ \\
        eIf{$type==1$}{
            $twthnode* k=twonode(m,node,pos \rightarrow left)$ \\
            $pos \rightarrow left=pos \rightarrow middle$ \\
            $pos \rightarrow middle=pos \rightarrow right$ \\
            eIf{$pos \rightarrow parent==NULL$}{
                $p \rightarrow m=pos \rightarrow d1$ \\
                $p \rightarrow n1=k$ \\
                $p \rightarrow n2=pos$
            } {
                $p=insert\_part(k,pos \rightarrow parent, pos \rightarrow d2,type)$
            }
            $pos \rightarrow d1=pos \rightarrow d2$ \\
        } {
            $twthnode* k=twonode(m,node,pos \rightarrow right)$ \\
            eIf{$pos \rightarrow parent==NULL$}{
                $p \rightarrow m=pos \rightarrow d2$ \\
                $p \rightarrow n1=pos$ \\
                $p \rightarrow n2=k$
            } {
                $p=insert\_part(k,pos \rightarrow parent, pos \rightarrow d2,type)$
            }
        }
            $pos \rightarrow right==NULL$ \\
            $pos \rightarrow d2= -1$
    }
    $return (p)$



\end{algorithm}

\begin{algorithm}
    \eIf{$pos \rightarrow type==0$} { 
        $free(pos)$ \\
        $pos=node$
    } {
        \eIf{$pos \rightarrow type==2$}{} {
            \eIf{$node \rightarrow type==0$}{} {
                $ret\_insert\_part* ret=insert\_part(node,pos,m,type)$ \\
                \eIf{$ret \rightarrow n1==NULL}{} {
                    $root=twonode(ret \rightarrow m,ret \rightarrow n1,ret \rightarrow n2)$
                }
            }
        }
    }
    $return$
\end{algorithm}
\subsection{Runtime Complexity Analysis}

\end{document}
