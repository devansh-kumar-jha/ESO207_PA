\documentclass[20pt]{article}
\usepackage{amsmath}
\usepackage{graphicx}
\usepackage{hyperref}
\usepackage[utf8]{inputenc}
\usepackage{polynom}
\usepackage[ruled, lined, linesnumbered, commentsnumbered, longend]{algorithm2e}
\usepackage{xcolor}

\title{ESO207 Programming Assignment 2.2}
\author{Devansh Kumar Jha(200318) and Divyansh Gupta(200351)}
\date{2021–10-30}
\begin{document}
\maketitle


\section{Data Structure Usage}
Two-Three tree ADT controlled using a class \textbf{twth} and structure \textbf{twthnode}.

\section{Strategy Used}


\newcommand\mycommfont[1]{\small\ttfamily\textcolor{blue}{#1}}
\SetCommentSty{mycommfont}

\section{Structure Used}
The structure used for the program \textbf{twthnode} is declared as follows -

\begin{algorithm}

	struct twthnode \{ \\
		\hspace{1cm} int type; \\
		\hspace{1cm} \tcc{0 null 1 single 2 leaf 3 twonode 4 threenode}
		\hspace{1cm} int d1,d2; \\
		\hspace{1cm} \tcc{2 values are stored and when not needed they are set to -1}
		\hspace{1cm} struct twthnode* parent; \\
		\hspace{1cm} struct twthnode* left; \\
		\hspace{1cm} struct twthnode* middle; \\
		\hspace{1cm} struct twthnode* right; \\
	\}

	\caption{Structure Declaration}

\end{algorithm}

\section{Pseudo Code}

\begin{itemize}
\item \textbf{Split(T,x)} \\
This function will split the given tree at
\item \textbf{Repair()} \\
\end{itemize}

\section{Runtime Complexity Analysis}

\end{document}
